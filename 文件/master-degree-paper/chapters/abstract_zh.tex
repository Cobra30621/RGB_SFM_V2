\documentclass[class=NCU_thesis, crop=false]{standalone}
\begin{document}

\chapter{摘要}

隨著深度學習技術在各領域的廣泛應用,可解釋性模型的重要性日益突顯。許多高效的深度學習模型雖然在預測準確度上表現出色,
但其"black-box"性質使得使用者難以理解其內部運作和決策過程。
在醫療、金融和軍事等需要高度可信賴和透明度的領域,
模型的可解釋性至關重要。
能夠解釋其決策過程的模型,不僅能增強使用者對模型的信任,還能在出現異常時提供有價值的洞見。

本研究提出了一種新型的基於卷積神經網路為基礎的新型可解釋性深度學習模型,
其目的在於提升模型的透明度和解釋能力。
該模型包括色彩感知區塊、輪廓感知區塊和特徵傳遞區塊三大部分。
色彩提取區塊透過計算輸入影像的不同部分的平均色彩與30種基礎調色色彩的相似度來提取輸入影像中不同部分的顏色特徵,
輪廓感知區塊則透過前處理將彩色影像變成灰階影像並計算輸入影像的不同部分與濾波器的相似度來檢測影像中的輪廓特徵,
色彩特徵傳遞區塊與輪廓特徵傳遞區塊則透過模擬視覺皮質的多階層架構方式,
在各自特徵輸入到特徵傳遞區塊的各層後會透過高斯卷積與特徵增強模組學習特徵,
並且將影像的特徵透過時序性合併的方式組成更完整的特徵輸出到下一層,
重複以上的步驟並逐層傳遞,
最後將輸出的色彩特徵與輪廓特徵結合後輸入進全連接層進行分類。
通過這些設計,使模型在保持高準確度的同時,提供清晰易懂的決策過程。

本研究一共使用了三種資料集分別是MNIST、Colored MNIST和Colored Fashion MNIST資料集,
通過以上三種資料集的實驗結果表明,
所提出的模型在可解釋性和性能方面均有不錯的表現。
尤其在Colored MNIST和Colored Fashion MNIST資料集上,
模型不僅能夠準確區分不同顏色和形狀的影像,
還能透過視覺化展示模型內部決策邏輯,
從而驗證其可解釋性和實用性。
這些結果證實了該模型在提升深度學習模型可解釋性方面的潛力和有效性。

\vspace{2em}
\noindent \textbf{關鍵字:} \keywordsZh{} % Set keywords in config.tex
\end{document}