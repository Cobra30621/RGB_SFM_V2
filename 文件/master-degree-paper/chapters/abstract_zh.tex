\documentclass[class=NCU_thesis, crop=false]{standalone}
\begin{document}

\chapter{摘要}

隨著深度學習技術在各領域的廣泛應用,可解釋性模型的重要性日益突顯。
現在有許多"black-box"模型雖然有高準確度但難以理解其決策過程,
而可解釋性模型,不僅能增強使用者對模型的信任,還能在出現異常時提供有價值的建議。

本研究提出了基於卷積神經網路的新型可解釋性深度學習模型,
該模型包括色彩感知區塊、輪廓感知區塊和特徵傳遞區塊三大部分。
色彩感知區塊透過計算輸入影像不同部分的平均色彩與30種基礎色彩的相似度來提取輸入影像的顏色特徵,
輪廓感知區塊則透過前處理將彩色影像變成灰階影像並輸入高斯卷積與特徵增強來檢測影像中的輪廓特徵,
特徵傳遞區塊則將輸入特徵進行高斯卷積與特徵增強後並且將
輸入特徵透過時序性合併的方式組成更完整的特徵輸出到下一層直到傳遞至全連接層,
最後將輸出的色彩特徵與輪廓特徵結合後輸入進全連接層進行分類。

本研究一共使用了三種資料集分別是MNIST、Colored MNIST和Colored Fashion MNIST資料集,
通過實驗結果表明,
本研究之模型在可解釋性和性能方面均有不錯的表現。
尤其在Colored MNIST和Colored Fashion MNIST資料集上,
模型不僅能夠準確區分不同顏色和形狀的影像,
還能透過視覺化展示模型內部決策邏輯,
從而驗證其可解釋性和實用性。

\vspace{2em}
\noindent \textbf{關鍵字:} \keywordsZh{} % Set keywords in config.tex
\end{document}